\documentclass[11pt,preprint, authoryear]{elsarticle}

\usepackage{lmodern}
%%%% My spacing
\usepackage{setspace}
\setstretch{1.2}
\DeclareMathSizes{12}{14}{10}{10}

% Wrap around which gives all figures included the [H] command, or places it "here". This can be tedious to code in Rmarkdown.
\usepackage{float}
\let\origfigure\figure
\let\endorigfigure\endfigure
\renewenvironment{figure}[1][2] {
    \expandafter\origfigure\expandafter[H]
} {
    \endorigfigure
}

\let\origtable\table
\let\endorigtable\endtable
\renewenvironment{table}[1][2] {
    \expandafter\origtable\expandafter[H]
} {
    \endorigtable
}


\usepackage{ifxetex,ifluatex}
\usepackage{fixltx2e} % provides \textsubscript
\ifnum 0\ifxetex 1\fi\ifluatex 1\fi=0 % if pdftex
  \usepackage[T1]{fontenc}
  \usepackage[utf8]{inputenc}
\else % if luatex or xelatex
  \ifxetex
    \usepackage{mathspec}
    \usepackage{xltxtra,xunicode}
  \else
    \usepackage{fontspec}
  \fi
  \defaultfontfeatures{Mapping=tex-text,Scale=MatchLowercase}
  \newcommand{\euro}{€}
\fi

\usepackage{amssymb, amsmath, amsthm, amsfonts}

\def\bibsection{\section*{References}} %%% Make "References" appear before bibliography


\usepackage[round]{natbib}

\usepackage{longtable}
\usepackage[margin=2.3cm,bottom=2cm,top=2.5cm, includefoot]{geometry}
\usepackage{fancyhdr}
\usepackage[bottom, hang, flushmargin]{footmisc}
\usepackage{graphicx}
\numberwithin{equation}{section}
\numberwithin{figure}{section}
\numberwithin{table}{section}
\setlength{\parindent}{0cm}
\setlength{\parskip}{1.3ex plus 0.5ex minus 0.3ex}
\usepackage{textcomp}
\renewcommand{\headrulewidth}{0.2pt}
\renewcommand{\footrulewidth}{0.3pt}

\usepackage{array}
\newcolumntype{x}[1]{>{\centering\arraybackslash\hspace{0pt}}p{#1}}

%%%%  Remove the "preprint submitted to" part. Don't worry about this either, it just looks better without it:
\makeatletter
\def\ps@pprintTitle{%
  \let\@oddhead\@empty
  \let\@evenhead\@empty
  \let\@oddfoot\@empty
  \let\@evenfoot\@oddfoot
}
\makeatother

 \def\tightlist{} % This allows for subbullets!

\usepackage{hyperref}
\hypersetup{breaklinks=true,
            bookmarks=true,
            colorlinks=true,
            citecolor=blue,
            urlcolor=blue,
            linkcolor=blue,
            pdfborder={0 0 0}}


% The following packages allow huxtable to work:
\usepackage{siunitx}
\usepackage{multirow}
\usepackage{hhline}
\usepackage{calc}
\usepackage{tabularx}
\usepackage{booktabs}
\usepackage{caption}


\newenvironment{columns}[1][]{}{}

\newenvironment{column}[1]{\begin{minipage}{#1}\ignorespaces}{%
\end{minipage}
\ifhmode\unskip\fi
\aftergroup\useignorespacesandallpars}

\def\useignorespacesandallpars#1\ignorespaces\fi{%
#1\fi\ignorespacesandallpars}

\makeatletter
\def\ignorespacesandallpars{%
  \@ifnextchar\par
    {\expandafter\ignorespacesandallpars\@gobble}%
    {}%
}
\makeatother

\newlength{\cslhangindent}
\setlength{\cslhangindent}{1.5em}
\newenvironment{CSLReferences}%
  {\setlength{\parindent}{0pt}%
  \everypar{\setlength{\hangindent}{\cslhangindent}}\ignorespaces}%
  {\par}


\urlstyle{same}  % don't use monospace font for urls
\setlength{\parindent}{0pt}
\setlength{\parskip}{6pt plus 2pt minus 1pt}
\setlength{\emergencystretch}{3em}  % prevent overfull lines
\setcounter{secnumdepth}{5}

%%% Use protect on footnotes to avoid problems with footnotes in titles
\let\rmarkdownfootnote\footnote%
\def\footnote{\protect\rmarkdownfootnote}
\IfFileExists{upquote.sty}{\usepackage{upquote}}{}

%%% Include extra packages specified by user

%%% Hard setting column skips for reports - this ensures greater consistency and control over the length settings in the document.
%% page layout
%% paragraphs
\setlength{\baselineskip}{12pt plus 0pt minus 0pt}
\setlength{\parskip}{12pt plus 0pt minus 0pt}
\setlength{\parindent}{0pt plus 0pt minus 0pt}
%% floats
\setlength{\floatsep}{12pt plus 0 pt minus 0pt}
\setlength{\textfloatsep}{20pt plus 0pt minus 0pt}
\setlength{\intextsep}{14pt plus 0pt minus 0pt}
\setlength{\dbltextfloatsep}{20pt plus 0pt minus 0pt}
\setlength{\dblfloatsep}{14pt plus 0pt minus 0pt}
%% maths
\setlength{\abovedisplayskip}{12pt plus 0pt minus 0pt}
\setlength{\belowdisplayskip}{12pt plus 0pt minus 0pt}
%% lists
\setlength{\topsep}{10pt plus 0pt minus 0pt}
\setlength{\partopsep}{3pt plus 0pt minus 0pt}
\setlength{\itemsep}{5pt plus 0pt minus 0pt}
\setlength{\labelsep}{8mm plus 0mm minus 0mm}
\setlength{\parsep}{\the\parskip}
\setlength{\listparindent}{\the\parindent}
%% verbatim
\setlength{\fboxsep}{5pt plus 0pt minus 0pt}



\begin{document}



\begin{frontmatter}  %

\title{Question2}

% Set to FALSE if wanting to remove title (for submission)




\author[Add1]{Joshua Strydom\footnote{\textbf{Contributions:}
  \newline \emph{The authors would like to thank no institution for
  money donated to this project. Thank you sincerely.}}}
\ead{20718284@sun.ac.za}





\address[Add1]{Stellenbosch University, Stellenbosch, South Africa}



\vspace{1cm}





\vspace{0.5cm}

\end{frontmatter}



%________________________
% Header and Footers
%%%%%%%%%%%%%%%%%%%%%%%%%%%%%%%%%
\pagestyle{fancy}
\chead{}
\rhead{}
\lfoot{}
\rfoot{\footnotesize Page \thepage}
\lhead{}
%\rfoot{\footnotesize Page \thepage } % "e.g. Page 2"
\cfoot{}

%\setlength\headheight{30pt}
%%%%%%%%%%%%%%%%%%%%%%%%%%%%%%%%%
%________________________

\headsep 35pt % So that header does not go over title




\hypertarget{introduction}{%
\section{\texorpdfstring{Introduction
\label{Introduction}}{Introduction }}\label{introduction}}

The weather in London over the period 1979 through 2020 was analysed. I,
the author, am to be of the opinion that the weather in London is not as
enjoyable as my female classmate thinks it is. Having analysed the data,
the weather in London could be said to be fair. London, on average,
experiences neither great weather nor terrible weather.

\begin{quote}
Figures \ref{Figure1}, \ref{Figure2}, \ref{Figure3} and \ref{Figure4}
make an attempt at finding out which months of the year are
characterized by different weather patterns. It is noted that the months
of June through September are arguably the most enjoyable months in
London.
\end{quote}

\begin{figure}[H]

{\centering \includegraphics{Question2_files/figure-latex/Figure1-1} 

}

\caption{Cloud cover \label{Figure1}}\label{fig:Figure1}
\end{figure}

As can be seen in \ref{Figure1}, cloud cover is the least during the
months of August and September. July and April are not far behind in
this regard. I consider less cloud cover to be an indicator of good
weather.

\begin{figure}[H]

{\centering \includegraphics{Question2_files/figure-latex/Figure2-1} 

}

\caption{Sunshine \label{Figure2}}\label{fig:Figure2}
\end{figure}

As can be seen in \ref{Figure2}, the months of June and July experience
the most amount of sunshine. May and August follow closely behind. I
consider more sunshine to be an indicator of good weather. The early and
late parts of each year tend to experience less sunshine on average.

\begin{figure}[H]

{\centering \includegraphics{Question2_files/figure-latex/Figure3-1} 

}

\caption{Mean temperature \label{Figure3}}\label{fig:Figure3}
\end{figure}

As can be seen in \ref{Figure3}, the months of August and July
experience the hottest average temperatures. June and September follow
closely behind. The hottest month, on average, barely reaches 20
degrees. In comparison to South Africa, London is a cold place to live.

\begin{figure}[H]

{\centering \includegraphics{Question2_files/figure-latex/Figure4-1} 

}

\caption{Mean precipitation \label{Figure4}}\label{fig:Figure4}
\end{figure}

As can be seen in \ref{Figure4}, the months of October and November
experience the most average precipitation. December and January follow
closely behind. It seems as though it is highly likely to a person
living in London will experience during all months throughout the year.
That is not to say that it is continuous rain though.

\begin{table}[H]
\centering
\begin{tabular}{rrrrr}
  \hline
 & Avecloudcover & Avesunshine & Avemeantemp & Aveprecipitation \\ 
  \hline
1 & 5.27 & 4.34 & 11.44 & 1.67 \\ 
   \hline
\end{tabular}
\caption{Average values throughout the year \label{tab1}} 
\end{table}

The yearly average of each factor can be seen in \ref{tab1}. London is
not a hot place to live as the city averages roughly 11.44 degrees
throughout the year. Cloud cover seems to be reasonably high as well. As
aforementioned, it is likely to rain every month of the year in London
but is is also likely to be reasonably sunny.

\hypertarget{conclusion}{%
\section{Conclusion}\label{conclusion}}

A persons preference for weather types is highly subjective. My personal
preference is toward a warm climate with sporadic rainfall. London has a
modest climate with frequent rainfall. All in all, the weather is not
awful and it is not mostly sunny nor mostly cold or rainy. The weather
is modest but is slightly more characterized by bad weather than good
weather.

\newpage

\bibliography{Tex/ref}





\end{document}
